\section{Linearizability: A Correctness Condition for Concurrent Objects\cite{HerlihyWing1990} }

\subsection{Motivation}

Concurrency is now a given rather than a design choice, and shared
\emph{concurrent objects} are unavoidable in system construction. The remaining
challenge is no longer how to exploit concurrency, but how to ascribe precise
meaning to object behavior in the presence of arbitrary interleavings. The challenge is how to
exploit concurrency and assign precise meaning to object behavior in the
presence of interleavings, that may grow astronomically. In the battle between
performance and correctness, the canons of sound engineering practices always
should prioritize correctness, thus motivating the study of correctness conditions
for concurrent objects.

The fundamental question concurrency raises is: what is the intended behavior of an
object when its operations are interleaved in many possible ways? If thread~A
performs $m$ operations and thread~B performs $n$ operations, the number of
possible interleavings grows combinatorially, on the order of
$\binom{m+n}{m}$. Whether such executions are correct depends entirely on the
object’s specification.

For example, a queue is not tied to a single behavior; instead it comes with a
specification that describes which input/output traces are allowed. In a classic FIFO,
queue, if elements are enqueued as $\langle 1,2,3\rangle$, every legal execution must
dequeue them as verbatim $\langle 1,2,3\rangle$. In contrast, large-scale asynchronous
systems such Kafka-based queues typically adopt a different delivery specification(s).
For example \emph{at-least-once delivery}, in such a system producing $\langle 1,2,3\rangle$
may legally result in consumer-visible traces like $\langle 1,1,1,2,3\rangle$ or multiple
replays of $\langle 1,2,3\rangle$  $\langle 1,2,3\rangle$. Correctness, therefore, is
a function of the chosen delivery specification.

This leads to a central question: who defines correctness, and how can it
be verified? Linearizability answers this question by providing a precise and
compositional correctness condition that allows programmers to reason about
concurrent executions using familiar sequential semantics.

\subsection{Definition of Linearizability}

A history $H$ is \emph{linearizable} if it can be extended to a history $H'$ such
that:
\begin{enumerate}
  \item $\mathrm{complete}(H')$ is equivalent to a legal sequential history $S$,
  \item the real-time order is preserved, i.e., $<_H \subseteq <_S$.
\end{enumerate}

Here, $\mathrm{complete}(H')$ denotes the maximal subsequence of $H'$ obtained by
removing pending invocations. Linearizability permits nondeterminism: multiple
sequential histories may justify the same concurrent execution, provided that at
least one such history exists.




\subsection{Proposed Solution}

Consider an execution history $H$ consisting of invocation and response events
observed over time. We assume that events are totally ordered by real time, in
the sense that no two events occur at exactly the same instant; any two events are
separated by some (possibly very small) time difference.

Such a history admits two complementary views. At the lowest level, $H$ reflects
the concrete execution of the program: the individual instructions that realize
each method call. We refer to this as the \emph{representation-level} view
(\textsf{REP}). At a higher level, the same execution can be viewed in terms of
method invocations and responses governed by an abstract object specification;
this is the \emph{abstract-level} view (\textsf{ABS}).

Linearizability asks whether the behavior observed at the representation level
can be explained by some legal abstract execution. Rather than reasoning directly
about all possible interleavings, the paper suggests a different perspective.
One may consider the set of all sequential executions obtained by linearizing
the concrete history $H\!\restriction_{\textsf{REP}}$, and ask whether each such
execution admits a valid abstract explanation.

Formally, suppose there exists an invariant $I$ over representation states and a
mapping
\[
A : \textsf{REP} \to 2^{\textsf{ABS}}
\]
that associates each concrete state with abstract state.

The abstraction function maps each concrete state to a set of abstract states
consistent with the specification. If, for all
$r \in \mathrm{Lin}(H\!\restriction_{\textsf{REP}})$, the invariant $I(r)$ holds
and
\[
  A(r) \subseteq \mathrm{Lin}(H\!\restriction_{\textsf{ABS}}),
\]
then the concrete implementation is linearizable with respect to the abstract
object.

This perspective, implicit in the original formulation of linearizability,
provides a natural bridge to later work that seeks to mechanize such reasoning
and to check these conditions automatically.
