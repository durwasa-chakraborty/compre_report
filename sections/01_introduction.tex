% ===============================================================================
% Introduction Section
% ===============================================================================
\section{Introduction}
In contemporary computing systems, performance improvements are achieved
primarily through the effective utilization of multicore architectures,
rather than through continued increases in single-core clock frequency. This
architectural shift follows the breakdown of Dennard scaling
\cite{Dennard1974}, under which further increases in transistor density could
no longer be accompanied by proportional reductions in power consumption and
heat dissipation. Consequently, processor design has evolved toward multicore
architectures as the principal mechanism for performance scaling.

As concurrency has become ubiquitous, the central challenge has shifted from
achieving parallelism to reasoning about correctness in its presence. The
nondeterminism inherent in concurrent execution, together with the exponential
growth of possible interleavings on multicore systems, renders conventional
testing methodologies that involves an array of testing suites inadequate.
Establishing correctness therefore requires formal reasoning frameworks capable
of expressing and verifying properties such as atomicity, linearizability, and progress.

The remainder of this report is organized as follows. Section 2 presents the
formal definition of linearizability\cite{HerlihyWing1990} . Section 3 surveys automated and
semi-automated approaches to linearizability verification, with particular
emphasis on mechanically assisted techniques.~\cite{Vafeiadis2010}.Building on
this background, Section 4 investigates linearizability in the
context of a non-trivial concurrent data structure, namely \emph{kCAS}.\cite{Guerraoui2013}
This section highlights why traditional linearization-point-based reasoning
becomes inadequate in these settings.To address these challenges, we then discuss
modern reasoning frameworks based on concurrent separation logic.
In particular, we examine recent advances such as Zoo, which provides a
compositional framework for verifying fine-grained concurrent algorithms,
as well as Cosmo, a concurrent separation logic tailored to the multicore
OCaml memory model. TODO.
