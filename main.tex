 \documentclass[sigplan]{acmart}

% ----------------------------
% Packages (keep minimal)
% ----------------------------
% Add this line BEFORE \usepackage{amssymb}
\let\Bbbk\relax
\usepackage{amssymb}
\usepackage{amsmath}
\usepackage{listings}
\usepackage{xcolor}
\usepackage{tikz}
\usepackage{mathpartir}
\usepackage{hyperref}
\usepackage{mathtools}

% Add this to your preamble for proper code listing styling6:


% ----------------------------
% Metadata (required by ACM)
% ----------------------------
\title{On the Limits of Automated Linearizability Verification for Modern Concurrency Patterns}
{\titlenote{The use of a prepositional opening, often reads as non-assertive. However, this is deliberate: the document is written in reported speech, and this stylistic choice is adopted from Adam Smith’s \emph{The Wealth of Nations}.}}
\author{Durwasa Chakraborty}
\affiliation{
  \institution{Indian Institute of Technology Madras}
  \department{Department of Computer Science}
  \country{India}
}
\email{cs24d011@cse.iitm.ac.in}

% Optional: remove if not needed
\settopmatter{
  printacmref=false,
  printccs=false,
  printfolios=false
}

\lstset{
  basicstyle=\small\ttfamily,
  breaklines=true,
  columns=fullflexible,
  xleftmargin=10pt,
  xrightmargin=10pt,
  frame=single,           % Frame ALL listings
  framesep=5pt,
  rulecolor=\color{black!20},  % Light gray border
}


\lstdefinestyle{Scala}{
  language=Scala,
  basicstyle=\ttfamily,
  keywordstyle=\bfseries\color{blue},
  commentstyle=\itshape\color{gray},
  stringstyle=\color{red},
  showstringspaces=false,
  breaklines=true
}

\renewcommand\footnotetextcopyrightpermission[1]{}
\acmConference[Preprint]
  {Technical Report}
  {2026}
  {Chennai, India}

% ----------------------------
% Document
% ----------------------------
\begin{document}


% ----------------------------
% Abstract
% ----------------------------
\begin{abstract}
  The transition to multicore architectures has made concurrent
programming a necessity rather than an optimization. While
linearizability provides a precise correctness condition for
concurrent objects, automatically verifying fine-grained lock-free algorithms
remains challenging due to interleavings, complex atomic
primitives, and weak memory behavior.

This report examines the progression from formal specifications
of concurrent correctness, to automated reasoning techniques,
to mechanized verification of realistic lock-free data structures,
and finally to abstractions that support compositional
concurrency under relaxed memory models. Through this layered
study, we highlight the central tension between expressive
concurrent algorithms and the rigor required to prove them
correct. The overarching theme is that scalable concurrency
demands not only sophisticated synchronization primitives, but
verification frameworks and abstractions capable of reasoning
across semantic, algorithmic, and memory-model boundaries.


\end{abstract}

\maketitle

% ----------------------------
% Sections
% ----------------------------

% ===============================================================================
% Introduction Section
% ===============================================================================
\section{Introduction}
In contemporary computing systems, performance improvements are achieved
primarily through the effective utilization of multicore architectures,
rather than through continued increases in single-core clock frequency. This
architectural shift follows the breakdown of Dennard scaling
\cite{Dennard1974}, under which further increases in transistor density could
no longer be accompanied by proportional reductions in power consumption and
heat dissipation. Consequently, processor design has evolved toward multicore
architectures as the principal mechanism for performance scaling.

As concurrency has become ubiquitous, the central challenge has shifted from
achieving parallelism to reasoning about correctness in its presence. The
nondeterminism inherent in concurrent execution, together with the exponential
growth of possible interleavings on multicore systems, renders conventional
testing methodologies that involves an array of testing suites inadequate.
Establishing correctness therefore requires formal reasoning frameworks capable
of expressing and verifying properties such as atomicity, linearizability, and progress.

This report is organized as follows. We begin by recalling the formal
definition of linearizability~\cite{HerlihyWing1990} in \S2 seminal work by Heerlihy and Wing, the canonical
correctness condition for concurrent objects.

\S3 surveys automated approaches to linearizability, focusing on
the work of Victor Vafeiadis~\cite{Vafeiadis2010}, which demonstrates how
linearization points and interference reasoning can be
mechanized.

We then study Zoo~\cite{zoo2026},\S4, by Allain Cl\'{e}ment and Gabriel Scherer, a framework for verifying concurrent
OCaml5 programs using separation logic. Zoo embeds a core fragment of
OCaml into Rocq and enables machine-checked proofs of fine-grained
concurrent data structures. To understand the practical implications of
such verification, we examine a data structure based on multi-word
compare-and-swap (kCAS)\cite{Guerraoui2013} by Rachid Guerraoui, Alex Kogan, Virendra J. Marathe and Igor Zablotchi, \S5 originally proposed in practical form by
Harris and Fraser~\cite{Harris2002}. Unlike single-word CAS, kCAS
supports atomic updates to multiple memory locations.

Zoo \cite{zoo2026} assumes sequential consistency, whereas real-world Multicore OCaml
operates under a relaxed memory model. To address this discrepancy, we
study Cosmo~\cite{cosmo2020}, \S6, by M\'{e}vel, Glen Jourdan, Jacques-Henri and Fran\c{c}ois Pottier, a concurrent separation logic tailored to
OCaml’s weak memory semantics. Cosmo reconciles high-level reasoning
principles with low-level memory behavior, exposing the additional
subtleties introduced by relaxed memory.

Finally, we consider Reagents~\cite{reagent2012}, \S7, by Aaron Turon, which propose a
transactional abstraction for composing fine-grained concurrent
interactions. Reagents structure lock-free algorithms around explicit
commit boundaries, offering a compositional model that resembles
transactional memory while retaining scalability. Through these successive acts, ie. from specification, to automation, to mechanization, to
memory-model awareness, and ultimately to compositional concurrency
the report explores how modern verification techniques engage with
increasingly expressive concurrent programming paradigms.


\section{Linearizability: A Correctness Condition for Concurrent Objects\cite{HerlihyWing1990} }\label{sec:linearizability}

\subsection{Motivation}

Concurrency is now a given rather than a design choice, and shared
\emph{concurrent objects} are unavoidable in system construction. The remaining
challenge is no longer how to exploit concurrency, but how to ascribe precise
meaning to object behavior in the presence of arbitrary interleavings. The
challenge is how to exploit concurrency and assign precise meaning to object
behavior in the presence of interleavings, that may grow astronomically. In the
battle between performance and correctness, the canons of sound engineering
practices always should prioritize correctness, thus motivating the study of
correctness conditions for concurrent objects.

The fundamental question concurrency raises is: what is the intended behavior of
an object when its operations are interleaved in many possible ways? If thread~A
performs $m$ operations and thread~B performs $n$ operations, the number of
possible interleavings grows combinatorially, on the order of $\binom{m+n}{m}$.
Whether such executions are correct depends on the object’s specification.


The term "FIFO queue" alone is imprecise. A proper specification formalizes the
data structure's behavior by explicitly characterizing the set of valid
input/output traces, rather than relying on informal descriptions. In a classic
FIFO, queue, if elements are enqueued as $\langle 1,2,3\rangle$, every legal
execution must dequeue them as verbatim $\langle 1,2,3\rangle$. In contrast,
large-scale asynchronous systems such Kafka-based queues typically adopt a
different delivery specification(s). For example \emph{at-least-once delivery},
in such a system producing $\langle 1,2,3\rangle$ may legally result in
consumer-visible traces like $\langle 1,1,1,2,3\rangle$ or multiple replays of
$\langle 1,2,3\rangle$ $\langle 1,2,3\rangle$. Correctness, therefore, is a
function of the chosen delivery specification.

This leads to a central question: who defines correctness, and how can it be
verified? Linearizability answers this question by providing a precise and
compositional correctness condition that allows programmers to reason about
concurrent executions using familiar sequential semantics.

\subsection{Definition of Linearizability}

A \emph{history} $(H)$ is a finite sequence of invocation and response events of
operations executed by a set of processes (or threads).

\begin{itemize}
  \item An \emph{invocation event} has the form $\mathsf{inv}\langle p,
    \mathit{op}, x \rangle $, meaning process $(p)$ invokes operation
    $(\mathit{op})$ with argument $(x)$.

  \item A \emph{response event} has the form $\mathsf{res}\langle p,
    \mathit{op}, y \rangle $, meaning process $(p)$ returns from operation
    $(\mathit{op})$ with result $(y)$.
\end{itemize}

A history need not be well-matched: it may contain \emph{pending invocations}
(i.e., invocations without corresponding responses).

An operation is \emph{complete} in $(H)$ if its invocation is followed by a
matching response in $(H)$.

A history $(H')$ is an \emph{extension} of a history $(H)$ if:

\begin{enumerate}
  \item $(H)$ is a prefix of $(H')$, and

  \item $(H')$ is obtained from $(H)$ by appending zero or more response events
    corresponding to pending invocations in $(H)$.
\end{enumerate}

Thus, extending a history may complete some pending operations or complete none
at all (a zero extension), but it does not introduce any new invocations.


A history $H$ is \emph{linearizable} if it can be extended to a history $H'$
such that:
\begin{enumerate}
  \item $\mathrm{complete}(H')$ is equivalent to a legal sequential history $S$,
  \item the real-time order is preserved, i.e., $<_H \subseteq <_S$.
\end{enumerate}


Here, $\mathrm{complete}(H')$ denotes the maximal subsequence of $H'$ obtained
by removing pending invocations. Linearizability permits nondeterminism:
multiple sequential histories may justify the same concurrent execution.



\subsection{Proposed Solution}
\begin{figure}[t]
  \centering
  \begin{tabular}{l}
    \hline \texttt{Enq} $= \mathbf{proc}$ $(q: \text{queue}, x: \text{item})$
    \\ \quad $i: \text{int} := \mathtt{INC}(q.\mathit{back})$ \quad \% Allocate
    a new slot. \\ \quad $\mathtt{STORE}(q.\mathit{items}[i], x)$ \quad \% Fill
    it. \\ $\mathbf{end}$ \texttt{Enq} \\ \\ \texttt{Deq} $= \mathbf{proc}$ $(q:
    \text{queue})$ $\mathbf{returns}$ $(\text{item})$ \\ \quad $\mathbf{while}$
    \texttt{true} $\mathbf{do}$ \\ \quad \quad $\mathit{range}: \text{int} :=
    \mathtt{READ}(q.\mathit{back}) - 1$ \\ \quad \quad $\mathbf{for}$ $i:
    \text{int}$ $\mathbf{in}$ $1 \ldots \mathit{range}$ $\mathbf{do}$ \\ \quad
    \quad \quad $x: \text{item} := \mathtt{SWAP}(q.\mathit{items}[i],
    \mathbf{null})$ \\ \quad \quad \quad $\mathbf{if}$ $x \neq \mathbf{null}$
    $\mathbf{then}$ $\mathbf{return}(x)$ $\mathbf{end}$ \\ \quad \quad
    $\mathbf{end}$ \\ \quad $\mathbf{end}$ \\ $\mathbf{end}$ \texttt{Deq}
    \\ \hline
  \end{tabular}
  \caption{Lock-free queue enqueue and dequeue operations.}
  \label{fig:queue-ops}
\end{figure}

To address the verification challenge, we consider the Herlihy-Wing queue (HW
hereinafter) as our primary case study. As shown in Figure \ref{fig:queue-ops},
the HW queue is implemented using an infinite array initialized with
$\mathbf{null}$. We distinguish between two levels of operation. The
\textbf{Abstract (ABS)} level comprises the high-level specification methods,
\texttt{Enq} and \texttt{Deq}, which define the external interface of the
object. The \textbf{Representation (REP)} level consists of the concrete, atomic
implementation steps (capitalized in the figure, e.g., \texttt{INC},
\texttt{STORE}, \texttt{SWAP}, \texttt{READ}) that compose the ABS methods.

Unlike a sequential specification where a concrete state $r$ maps to a single
abstract state $q$ (a function $REP \to ABS$), a concurrent implementation
allows for ambiguity. Due to the overlap of concurrent operations, the concrete
memory state may effectively represent multiple valid abstract states
simultaneously. To capture this, we define our abstraction function $A(r)$ to
map a concrete state to the power set of abstract states:
$$A(r): REP \to 2^{ABS}$$ This relaxation allows us to model all possible
combinations of valid states that the system could be in, accounting for the
non-determinism inherent in concurrency.

In terms of the concrete implementation, if a \texttt{STORE} operation that
writes $x$ to the array completes before an \texttt{INC} operation allocates a
slot for $y$ , then $x$ is guaranteed to occupy a lower array index than $y$,
establishing the order $x<_ry$.

Using this partial order, we define the abstraction function $A(r)$ as the set
of all abstract queues $q$ that satisfy two conditions. First, \emph{Content
Preservation} requires that the items in the queue match the items in the array
($items(q) = items(r)$). Second, \emph{Order Preservation} requires that the
total order of the queue $<_q$ is consistent with the partial order of the array
$<_r$; that is, $<_r \subseteq <_q$. Formally:
$$A(r) = \{ q \mid items(q) = items(r) \wedge <_r \subseteq <_q \}$$ This
definition captures the essence of the HW queue: while items are ordered by
index, the concurrent nature of the \texttt{Enq} method (specifically the gap
between \texttt{INC} and \texttt{STORE}) means that the ``logical'' order of
enqueues may vary until the values are physically visible in the array.

The ultimate goal is to prove that the implementation is linearizable. We recall
that a history $H$ is linearizable if it can be extended to a complete history
$H'$ that is equivalent to a legal sequential history $S$. In our
set-theoretical model, we denote the set of all such legal sequential histories
with the current execution as $Lin(H|_{ABS})$.
$$A(r) \subseteq Lin(H|_{ABS})$$ If this condition holds initially and is
preserved by every atomic step (\texttt{REP} transition) of the algorithm, we
successfully can say that the program order respects the correctness
specification.

\begin{figure}[t]
\centering \footnotesize \renewcommand{\arraystretch}{1.05}
\begin{tabular}{|l|c|c|c|}
\hline \textbf{Event} & $A(Lin(H|_{\tiny REP}))$ & $Lin(H|_{\tiny ABS})$ & V
\\ \hline Init & $\{[]\}$ & $\{[]\}$ & $\checkmark$ \\ \hline A: ENQ(X) INV &
$\{[]\}$ & $\{[],[x]\}$ & $\checkmark$ \\ \hline A: INC & $\{[]\}$ &
$\{[],[x]\}$ & $\checkmark$ \\ \hline \colorbox{yellow!25}{A: STORE(X)} &
\colorbox{yellow!25}{$\{[],[x]\}$} & \colorbox{yellow!25}{$\{[],[x]\}$} &
\colorbox{yellow!25}{$\checkmark$} \\ \hline A: OK & $\{[x]\}$ & $\{[],[x]\}$ &
$\checkmark$ \\ \hline B: ENQ(Y) INV & $\{[x]\}$ & $\{[],[x],[y],[x,y],[y,x]\}$
& $\checkmark$ \\ \hline B: INC & $\{[x]\}$ & $\{[],[x],[y],[x,y],[y,x]\}$ &
$\checkmark$ \\ \hline \colorbox{yellow!25}{B: STORE(Y)} &
\colorbox{yellow!25}{$\{[x],[x,y]\}$} &
\colorbox{yellow!25}{$\{[],[x],[y],[x,y],[y,x]\}$} &
\colorbox{yellow!25}{$\checkmark$} \\ \hline B: OK & $\{[x,y]\}$ & $\{[x,y]\}$ &
$\checkmark$ \\ \hline
\end{tabular}
\caption{Verification trace of concurrent enqueues}
\label{fig:verification-trace}
\end{figure}



This trace, as demonstrated in Figure~\ref{fig:verification-trace}, illustrates
the verification process. We observe two concurrent processes: process A
enqueuing $x$ and process B enqueuing $y$. Initially, both the concrete array
and the set of possible abstract states are empty. When A invokes \texttt{ENQ}
and executes \texttt{INC}, the concrete state remains empty (since
\texttt{STORE} has not yet occurred), yet the abstraction set expands to include
both the empty queue and a queue containing $x$, reflecting the ambiguity of
whether A's write is yet visible. The critical moment arrives at A's
\texttt{STORE(X)}: the value now physically appears in the array, and the
abstraction set narrows from ${ [ ], [x] }$ to reflect this certainty. Before
B's \texttt{STORE(Y)}, the abstraction set ${[x]}$ contains only queues with $x$
present; after B's \texttt{STORE(Y)}, it expands to ${[x], [x,y]}$ because the
concrete memory now shows both values, but we cannot yet determine the relative
order imposed by external observers. The final step, when both operations
complete, converges to a single linearizable history: the queue $[x,y]$.
Throughout this execution, the invariant $A(Lin(H|*{\text{REP}})) \subseteq
Lin(H|*{\text{ABS}})$ is maintained, confirming that every state the concrete
implementation produces is a subset of the abstract specification.

\subsection{Why Linearizability is the Standard Correctness Condition}

Linearizability has become the standard correctness condition for concurrent
objects because it simultaneously preserves real-time semantics, supports
modular reasoning. To understand its importance, it is useful to compare it with
two closely related conditions: sequential consistency and serializability.

\subsection{Sequential Consistency}

Sequential consistency requires that a concurrent history be equivalent to some
legal sequential history, but it does \emph{not} require preservation of
real-time order. That is, if operation $e_1$ completes before $e_2$ begins in
the actual execution, sequential consistency does not require $e_1$ to precede
$e_2$ in the sequential explanation.

Consider a FIFO queue with the following history:

\[
\texttt{ENQ}(x)_A;\ \texttt{OK}_A;\ \texttt{ENQ}(y)_B;\ \texttt{OK}_B;\ \texttt{DEQ}()_B;\ \texttt{OK}(y)_B.
\]

Here, $x$ is clearly enqueued before $y$, yet $y$ is dequeued first. This
history can be made sequentially consistent by reordering the operations as if
$\texttt{ENQ}(y)$ occurred before $\texttt{ENQ}(x)$. However, this violates the
real-time order observed by an external observer. Linearizability forbids such
reordering because it requires that the sequential explanation respect the
real-time precedence relation.

More importantly, sequential consistency is \emph{not compositional}. Even if
each object is sequentially consistent in isolation, the whole system may not
be. Herlihy and Wing provide an example with two queue objects $p$ and $q$ where
each subhistory is sequentially consistent, but their composition is not. This
failure of locality means that objects cannot be verified independently, which
severely limits modular design.

\subsection{Serializability}

Serializability requires that concurrent \emph{transactions} be equivalent to
some sequential execution. In strict serializability, the sequential order must
also respect real-time order.

Serializability is \emph{also} not local. Even if each object’s projection of a
history is serializable, their combination may not be. Herlihy and Wing give a
history involving two objects $p$ and $q$ where each object individually admits
a serial explanation, yet no global serial order exists. Thus, serializability
does not compose.

Also, serializability is inherently \emph{blocking}. Consider two registers $x$
and $y$ and two transactions:

\[
A: \texttt{Read}(x); \texttt{Write}(y) \\ B: \texttt{Read}(y); \texttt{Write}(x)
\]

If both reads return $0$, neither write can complete without violating
serializability. One transaction must abort or wait.

By contrast, linearizability is nonblocking: a pending invocation of a total
operation can always be completed without waiting for another operation. This
makes it especially appropriate for low-level concurrent data structures and
multiprocessor systems.

\subsection{Compositionality: The Key Distinction}

The defining advantage of linearizability is \emph{locality}:

\[
H \text{ is linearizable } \iff \forall x.\ H|_x \text{ is linearizable}.
\]

Thus objects can be verified independently and then safely composed.



\subsection{Conclusion}

We have established a formal definition of linearizability and the mathematical
underpinnings required to prove a piece of code is linearizable. Practical
challenges remain, for example, deriving the precise abstraction function $A(r)$
for complex data structure seems to be a non-trivial task and requires some
ingenuity. However, these open questions do not diminish the foundational work
by Herlihy and Wing. This paper provides the bedrock to successfully talk about
\emph{correctness} from an intuition into a provable property.

\section{Automatically Proving Linearizability \cite{Vafeiadis2010}}\label{sec:cave}

\subsection{Motivation}

We discussed in the previous section that \emph{linearizability} is the standard
correctness condition for concurrent objects, yet proving linearizabilty for
realistic implementations remains a significant challenge. We explore an
alternate way of proving linearizability by identifying \emph{linearization
points}: an instant during the operations's execution at which the effect is
deemed to occur atomically. While this approach is effective for simple
implementations, it becomes increasingly difficult as algorithms grow more
sophisticated because of complex control flow or inter-thread interference. In
many cases, a linearization point may be conditional, may occur in a different
thread, or may depend on future execution. This is particularly common for
operations that do not logically modify the shared abstract state, such as
unsuccessful lookups or dequeue operations on an empty deque. For such
operations, insisting on a concrete linearization point is marked with certain
difficulty as it maybe in different places for different outputs.

Consider a lock-free stack with a \texttt{pop} operation.
\begin{itemize}
\item Scenario A (Success): If the stack is not empty, the linearization point
  is the successful \texttt{CAS} instruction that updates the head pointer.
\item Scenario B (Empty): If the stack is empty, the \texttt{CAS} is never
  attempted. The linearization point is instead the \texttt{READ} instruction
  where the thread observed the null head.
\end{itemize}

At the same time a different category of operations exist, \emph{effectful
operations}, different in the sense that they necessarily update the abstract
state, and those that are \emph{pure} which do not update the state. Effectful
operations often, by established definition of the point where the effect takes
place, becomes a natural candidate for linearization points, while pure
operations do not. The key insight of the paper is that these two classes of
executions should be treated differently, and that linearizability can be
established without explicitly committing to a linearization point for every
operation.

\subsection{Proposed Solution}

The proposed approach replaces reasoning about linearization points and
sequential histories with an instrumentation-based verification strategy. The
method augments the abstract specification directly into the concrete
implementation through additional state and assertions. Linearizability is then
reduced to the problem of showing that certain assertions cannot fail.

The core idea is to distinguish whether an operation has performed an effectful
abstract update or whether it has remained pure. For effectful executions, the
abstract operation is executed at a candidate instruction that is already
present in the implementation, such as a successful compare-and-swap. For pure
executions, the method refrains from choosing a specific linearization point;
instead, it records which return values are consistent with the abstract
specification at some point during the execution.

This distinction is realized by instrumenting each operation with auxiliary
variables that track abstract effects. One variable records the result of an
effectful abstract execution, if such an execution occurs. In parallel, an array
of flags records whether particular return values are admissible for a pure
execution according to the specification. Initially, no abstract effect is
assumed to have taken place, and no return value is assumed to be valid.

As the concrete execution proceeds, candidate effectful linearization points
trigger the execution of the abstract specification, and the result of this
execution is stored. Independently, whenever the specification permits a pure
operation to return a certain value without modifying the abstract state, the
corresponding admissibility flag is enabled. At each return point of the
concrete method, a single assertion is checked: either an effectful abstract
update has occurred and the concrete return value matches the recorded abstract
result, or no effectful update has occurred and the returned value is among
those permitted by the specification.


\begin{figure}[t]
\centering
\begin{lstlisting}[language=C]
int tryDequeue(void) { Node next, head, tail; int pval;

    while (true) { head = Q->head; tail = Q->tail; next = head->tl;

        if (Q->head != head) continue;

        if (head == tail) { if (next == NULL) return EMPTY;

            CAS(&Q->tail, tail, next); } else { pval = next->val; if
          (CAS(&Q->head, head, next)) return pval; } } }
\end{lstlisting}
\caption{Michael \& Scott queue tryDequeue (adapted from
  Vafeiadis~\cite{Vafeiadis2010}).}
\label{fig:ms-dequeue}
\end{figure}

\subsection{Example: Concurrent Queue}

The intuition behind the method can be illustrated using a concurrent queue with
operations \textit{enqueue} and \textit{dequeue}. Successful enqueue operations
are effectful: they necessarily extend the abstract queue. In typical
implementations, this effect corresponds to a concrete instruction that links a
new node into the data structure. At this instruction, the abstract enqueue
operation is executed, and its result is recorded.

In contrast, a dequeue, from Figure \ref{fig:ms-dequeue}, operation that returns
\textsf{EMPTY} does not modify the abstract queue. Such an execution may span
multiple reads and checks, and no single instruction correspond to its logical
effect. Rather than assigning it an arbitary linearization point, the method
simply checks whether returning \textsf{EMPTY} is consistent with the abstract
queue being empty at some point during the execution. If so, the return is
deemed admissible.

Determining the linearization point requires distinguishing between varying
execution states: whether the queue is empty, whether it is non-empty, and
whether the thread is actively "helping" another operation complete. Crucially,
this results in a linearization point that is both input-dependent and
execution-dependent. For example, if the queue is empty, the linearization point
is not a state-modifying CAS, but rather the atomic read confirming \texttt{next
  == NULL}. Thus, unlike enqueue, the \texttt{tryDequeue} method resists a
single static definition, requiring a case-by-case evaluation of the runtime
execution path.


\subsection{Verification Procedure}

\begin{figure}[t]
\centering
\begin{lstlisting}[language=C]
int tryDequeue(void) { Node next, head, tail; int pval;

    // --- instrumentation at entry --- lres = UNDEF; can_return[EMPTY] = false;

    while (true) { head = Q->head; tail = Q->tail; next = head->tl;

        if (Q->head != head) continue;

        // PURE CHECKER (symbolic) // if (AQ == empty) can_return[EMPTY] = true;

        if (head == tail) { if (next == NULL) {

                assert( lres == EMPTY || (lres == UNDEF && can_return[EMPTY]) );

                return EMPTY; }

            // --- CAS on tail if (CAS(&Q->tail, tail, next)) { assert (lres
              ==UNDEF); lres = spec_tryDequeue(); // WRONG LP, but still a
              candidate }

        } else { pval = next->val;

            // --- CAS on head if (CAS(&Q->head, head, next)) { assert(lres ==
              UNDEF); lres = spec_tryDequeue(); assert(lres == pval); return
              pval; } } } }
\end{lstlisting}
\caption{Instrumented \texttt{tryDequeue}}
\label{fig:ms-instrumented}
\end{figure}

The verification procedure presented in this work is an automatic technique for
proving the linearizability of concurrent data structure implementations. The
procedure operates by distinguishing between \emph{effectful} executions (which
modify the shared abstract state) and \emph{pure} executions (which do not). The
algorithm, formally defined as \texttt{PROVELINEARIZABLE}, takes as input the
library's constructor, its operations, and their corresponding atomic functional
specifications. The procedure is divided into two distinct phases: a
\emph{Preparation Phase} that generates auxiliary constructs and a
\emph{Checking Phase} that instruments the code for verification.

\paragraph{Pure Checker Generation.}
For each operation, a ``Pure Linearizability Checker'' is generated. This
checker is derived from the specification by identifying all syntactically pure
execution paths---paths that do not assign to global variables. Along these
paths, the return statement \texttt{return v} is replaced with the assignment
\texttt{can\_return[v] = true}.

\paragraph{Candidate Linearization Points.}
The function \\ \texttt{GETCANDIDATELINPOINTS} analyzes the source code to
identify potential effectful linearization points. It unfolds definitions of
atomic primitives (like CAS) to expose control flow and selects one
state-modifying command along each execution path as a candidate linearization
point.

\paragraph{Instrumentation Variables.}
In the checking phase, each operation $op_i$ is instrumented with two auxiliary
variables per thread. The variable \texttt{lres} stores the result of the
abstract operation if an effectful linearization point occurs; it is initialized
to \texttt{UNDEF}. The boolean array \texttt{can\_return}, indexed by possible
return values, tracks whether a valid pure linearization point has been
observed; all entries are initialized to \texttt{false}.

\paragraph{Code Transformation.}
The source code is modified at three places. At every identified effectful
linearization point, code is inserted to execute the abstract specification, and
an assertion \texttt{assert(lres == UNDEF)} ensures the linearization point
executes at most once per path. In the background, the verification tool
simulates the execution of the Pure Checker after every atomic command of the
current thread and after every step of concurrently executing threads, updating
\texttt{can\_return} continuously based on the state of the abstract object.
Finally, at each return point yielding value \texttt{res}, the following
assertion is inserted:
%
\begin{multline*}
  \texttt{assert}\bigl(lres = res \;\lor \\ (lres = \texttt{UNDEF} \land
  \textit{can\_return}[res])\bigr)
\end{multline*}
%
This asserts that either the operation linearized effectively (matching
\texttt{lres}) or it found a valid pure linearization point (tracked by
\texttt{can\_return}). If the verification tool establishes that all inserted
assertions hold on every execution path, the implementation is concluded to be
linearizable. Figure~\ref{fig:ms-instrumented} presents the instrumented version
of the code, where each \emph{CAS} instruction is augmented with assertion-based
instrumentation to capture potential effectful linearization points and a pure
linearization checker updates \texttt{can\_return} for the empty-queue case.

% ---------------------------------------------------------------

\subsection{{\sc Verify} and RGSep}


After the implementation has been transformed and instrumented with candidate
linearisation points and pure-checker assertions, the {\sc Verify} procedure is
invoked as the core proof engine \cite{vafeiadis2010rgsep}. {\sc Verify}
constructs a \emph{most-general client}: a model of an unbounded number of
threads each repeatedly calling the library methods. It then runs an abstract
interpretation over this model to establish two properties simultaneously: (i)
memory safety is maintained throughout execution, and (ii) none of the
instrumented assertions are ever violated. Both obligations are discharged using
the RGSep action-inference algorithm \cite{vafeiadis2010rgsep}. At a very high
level a thread's \emph{guarantee} ($G$) is the set of actions it promises to
restrict itself to when modifying shared memory, while its \emph{rely} ($R$) is
the set of actions it assumes the environment (all other threads) might perform
during its execution.

The reason why we need Rely-Guarantee is because standard Hoare logic,
$\{P\}\,c\,\{Q\}$, is unsound for concurrent programs: a postcondition
established by one thread may be invalidated immediately by an interfering
thread before the next instruction executes. RGSep \cite{vafeiadis2010rgsep}
addresses this by confining interference to the shared heap and governing it
entirely through actions, so that the verifier only reasons about one thread at
a time against a single \emph{global rely} bucket rather than enumerating all
possible interleavings.

\subsection{Worked Example: Inferring Guarantees for the M\&S Queue}

To make the above concrete, consider the Michael-Scott lock-free queue. The tool
infers the thread guarantee $G$ automatically by scanning the code for
successful compare-and-swap operations.

\paragraph{Action $G_1$ :: appending the new node.}
The first successful \texttt{CAS(\&tail->tl, next, node)} is identified.
Inspecting the pre- and post-states of the shared heap yields:
%
\begin{equation}
  G_1 \;\triangleq\; \texttt{tail}{\to}\texttt{tl} \mapsto \texttt{NULL}
  \;\leadsto\; \texttt{tail}{\to}\texttt{tl} \mapsto \texttt{node}.
\end{equation}
%
ie a thread may advance the \texttt{tl} pointer of the current tail node from
\texttt{NULL} to the freshly allocated \texttt{node}.

\paragraph{Action $G_2$ :: advancing the tail pointer.}
The second successful \texttt{CAS(\&Q->tail, tail, node)} yields:
%
\begin{equation}
  G_2 \;\triangleq\; \texttt{Q}{\to}\texttt{tail} \mapsto \texttt{tail}
  \;\leadsto\; \texttt{Q}{\to}\texttt{tail} \mapsto \texttt{node}.
\end{equation}
%
This captures the swing of the global tail pointer from the old tail cell to the
new one.

\paragraph{Composing the guarantee.}
The two discovered mutations are packaged into the thread's full guarantee:
%
\begin{equation}
  G \;=\; G_1 \cup G_2.
\end{equation}

All guarantees are collected into a single \emph{global rely},
$R_{\mathrm{global}} = \bigcup_i G_i$. At each atomic step of the thread under
verification, the verifier pauses at the current abstract shared state $S$ and
applies $R_{\mathrm{global}}$ to $S$, effectively simulating the environment
non-deterministically firing any action from the bucket.

At this point, one might naturally wonder: if we keep applying these operations,
won't the possibilities just keep expanding? If every application of the global
rely yields a new state that we then have to account for, it seems like the
state space would grow infinitely, making it impossible to ever reach a fixed
point.

The original paper addresses this exact issue using a \textsc{Stabilize} method,
which guarantees that a fixed point is indeed eventually achieved
\cite{vafeiadis2010rgsep}. However, I am omitting the formal details of
\textsc{Stabilize} from this report, as its underlying proofs are quite dense,
and I admittedly did not understand them well enough to confidently explain them
here.

Regardless of the underlying math the high-level takeaway is: because the
$R_{\mathrm{global}}$ bucket already contains \emph{every} action any thread
could ever perform, verifying a single thread against this one bucket is
equivalent to verifying it against an unbounded, arbitrary interleaving of all
threads. The state-explosion problem is thereby avoided entirely.


\subsection{Conclusion}
The approach presented in this paper offers a practical and conceptually clean
path to automated linearizability proofs. By separating effectful and pure
executions and embedding the abstract specification directly into the
implementation, it avoids the need for explicit linearization points in cases
where they are difficult to identify. Linearizability is reduced to a local
safety property, enabling the use of mature verification tools.

The method has been implemented in the \textsc{Cave} verification tool and
evaluated on a range of concurrent stacks, queues, and set implementations. The
results demonstrate that the approach is expressive enough to handle realistic
concurrent data structures while remaining amenable to automation.

\section{Zoo: A Framework for the Verification of Concurrent OCaml 5 Programs using Separation Logic\cite{zoo2026}}\label{sec:zoo}
\subsection{Motivation}

So far, we have focused on linearizability for data structures in which the
linearization point can be identified either within the operation itself or
within another method invoked during its execution. We now shift attention to
concurrent libraries operating under a language- specific memory model under
sequential consistency. The release of OCaml 5 introduced true parallelism via
multicore support, enabling a growing ecosystem of concurrent libraries such as
\textsf{Saturn}, \textsf{Eio}, and \textsf{Kcas}. While this substantially
broadened OCaml’s applicability, it also raised new verification challenges,
particularly the lack of a practical framework for reasoning about realistic
concurrent OCaml programs that rely on features such as algebraic data types and
mutable records.

Existing Iris \cite{iris-proofmode} based approaches, including \emph{HeapLang},
are expressive but poorly aligned with OCaml. They lack essential language
constructs, most notably algebraic data types and mutually recursive functions,
and require substantial manual translation from OCaml into the modeling
language. This translation burden complicates proof maintenance and weakens the
connection to source programs. The absence of atomic record fields forced
programmers to rely on unsafe encodings, revealing a mismatch between the
language’s abstractions and its intended concurrent usage.

\begin{figure}[t]
\centering
\begin{lstlisting}[language=OCaml]
(* Intended: atomic field inside a record *) type node = { value : int; next :
    node option }

(* Unsafe encoding to simulate atomic field *) let cas_next (n : node) oldv newv
  = let atomic_field = (Obj.magic (&n.next) : node option Atomic.t) in
  Atomic.compare_and_set atomic_field oldv newv
\end{lstlisting}
\caption{Simulating an atomic record field using an unsafe cast.}
\label{fig:unsafe-atomic-field}
\end{figure}

Figure~\ref{fig:unsafe-atomic-field} illustrates this issue. The programmer
wishes to treat the record field \texttt{next} as atomic, but since the type
system does not permit atomic annotations on fields, the code casts the field’s
memory to \texttt{Atomic.t} using \texttt{Obj.magic}. At this point, the type
system’s guarantees are circumvented: the field is declared nonatomic in the
type definition.

The consequence is that the semantic classification of the location (as atomic
or nonatomic) no longer aligns with the static program structure. This breaks
the abstraction boundary enforced by the type system and complicates formal
reasoning, since the memory model and verification logic must now account for
behavior that is invisible at the type level.


The paper is motivated by these limitations. Its goal is to develop a practical
verification framework for real-world concurrent OCaml programs, while refining
the language and its semantics to better support safe and verifiable programs.

\subsection{Proposed Solution}
The authors propose Zoo, a comprehensive framework for verifying concurrent
OCaml 5 programs using Iris, a state-of-the-art concurrent separation logic
mechanized in the Rocq proof assistant.
\vspace{-0.5\baselineskip}
\paragraph{ZooLang.}
At the core of Zoo is ZooLang, a language designed to faithfully model a
substantial fragment of OCaml. ZooLang supports algebraic data types, mutable
and immutable records, references, atomic operations, mutual recursion, and
concurrency primitives. It is deeply embedded in Rocq and comes with a formally
defined operational semantics and a corresponding Iris-based program logic. The
framework includes a tool, \textsf{ocaml2zoo}, which translates OCaml source
programs into ZooLang code embedded in Rocq. Unlike HeapLang, which introduces
various encodings in the translation that make the relation between source and
verified programs difficult to maintain, ZooLang is syntactically very close to
OCaml, ensuring that verification artifacts remain aligned with real
implementations.

\vspace{-0.5\baselineskip}
\paragraph{Specifications and Proofs.}
Once translated to ZooLang, users can write specifications and prove them in
Iris. For example, the specification of \texttt{stack\_push} is:

$\textsf{Lemma stack\_push\_spec t}\ \ell\ \textsf{v} : \\ \text{}\qquad<<<
\textsf{stack\_inv t} \ \ell \\ \text{}\qquad\qquad\mid \forall\forall
\ \textsf{vs},\ \text{stack\_model t vs} >>>
\\ \text{}\qquad\qquad\textsf{stack\_push t v}\ @\ \uparrow \ell
\\ \text{}\qquad<<< \textsf{stack\_model t (v :: vs)} \\ \text{}\qquad\qquad\mid
\textsf{RET (); True} >>>.$

As in Hoare logic\cite{hoare-logic}, the specification consists of pre- and
postconditions, each split into a private part (the stack invariant
\textsf{stack\_inv t}) and an atomic part (the abstract stack state
\textsf{stack\_model}). The atomic conditions capture the linearization point,
stating that during execution the stack’s abstract state is atomically updated
from \texttt{vs} to \texttt{v :: vs}.

%% (TODO Reduce the line space)

\vspace{-0.5\baselineskip}
\paragraph{Physical Equality: A Precise Semantics.}
A major technical contribution is a new, precise semantics for physical equality
and \texttt{compare\_and\_set}. Physical equality is essential for lock-free
algorithms but is dependent on compiler optimizations such as sharing and
unsharing. The authors show that equating physical equality with structural
equality is unsound. Instead, they model it as non-deterministic with carefully
specified guarantees. They also introduce a mechanism for controlling unsharing
through generative constructors. OCaml's physical equality operator
(\texttt{==}) presents a fundamental challenge for concurrent reasoning: it is
under-specified by the language semantics. The Zoo paper addresses this head-on
by introducing generative constructors, which restore stable identity properties
needed for sound CAS reasoning.

Physical equality in OCaml is not fully determined by the abstract value
semantics. For example:
\begin{verbatim}
  Some 0 == Some 0
\end{verbatim}
may return \texttt{true} or \texttt{false} depending on whether the two blocks
are represented identically in memory. Zoo formalizes this by distinguishing:
\begin{itemize}
  \item $\hat{v}_1 \mathrel{==} \hat{v}_2$: the values \emph{must} be physically
    equal,
  \item $\hat{v}_1 \approx \hat{v}_2$: the values \emph{may} be physically
    equal.
\end{itemize}
This distinction allows the semantics to avoid baking in false assumptions about
representation sharing, acknowledging that the compiler may make
implementation-specific choices about block allocation and layout.

\begin{figure}[t]
\centering \footnotesize

\begin{tabular}{c c}
\textbf{Source Program} & \textbf{Heap Layout} \\[6pt]

\begin{minipage}{0.45\linewidth}
\begin{verbatim}
(* Before optimization *) let x = Some 0 in let y = x in x == y (* true *)
\end{verbatim}
\end{minipage}
&
\begin{minipage}{0.45\linewidth}
Before unsharing:

x ----\textbackslash \\ > [Some | 0] (B1) \\ y ----/
\end{minipage}

\\[12pt]

\begin{minipage}{0.45\linewidth}
\begin{verbatim}
(* After compiler unsharing *) let x = Some 0 in let y = Some 0 in x == y (*
 false *)
\end{verbatim}
\end{minipage}
&
\begin{minipage}{0.45\linewidth}
After unsharing:

x ----> [Some | 0] (B1) \\ y ----> [Some | 0] (B2) \\
\end{minipage}

\end{tabular}

\caption{Unsharing of immutable blocks in OCaml. The abstract value \texttt{Some
    0} remains the same, but physical identity changes due to compiler
  duplication.}
\label{fig:unsharing}
\end{figure}

As shown in Figure~\ref{fig:unsharing}, the original program binds \texttt{y} to
\texttt{x}, so both variables reference the same heap block. Since OCaml’s
\texttt{==} operator checks physical equality (pointer identity), \texttt{x ==
  y} evaluates to \texttt{true}.

After compiler unsharing, the expression \texttt{Some 0} is duplicated. Although
the abstract values are structurally equal, they now occupy distinct heap
blocks. Consequently, \texttt{x == y} evaluates to \texttt{false}.

This phenomenon is semantically invisible under structural equality but
observable under physical equality. In concurrent code, where algorithms
sometimes rely on pointer identity for synchronization, such unsharing can
introduce correctness bugs.


\subsection{CAS Vulnerability to Unsharing}

Compare-and-swap (CAS) operations rely on physical equality:
\begin{verbatim}
  Atomic.compare_and_set loc expected new
\end{verbatim}
succeeds only if the current value at \texttt{loc} is physically identical to
\texttt{expected}.

If the compiler unshares \texttt{expected}, CAS behavior can change
unexpectedly. Initially, when \texttt{expected} and the value at \texttt{loc}
reference the same block, CAS succeeds. After unsharing, \texttt{expected}
becomes a distinct copy, so CAS fails despite abstract equality. Thus, program
behavior may vary depending on compiler optimizations, making concurrent
reasoning fragile.


Zoo introduces \emph{generative immutable blocks}, marked with the
\texttt{[@generative]} annotation.


For generative blocks, physical equality is tightly specified:
\[
\hat{v}_1 \mathrel{==} \hat{v}_2 \iff \mathit{bid}_1 = \mathit{bid}_2
\]
This makes them behave like mutable references in terms of identity, even though
their contents are immutable.

The key invariant now is \emph{the compiler must not unshare generative blocks}.

When a constructor is marked \texttt{[@generative]} the compiler preserves the
block's identity across the entire computation. Physical equality becomes
deterministic with respect to that block and we enforce two generative blocks
are physically equal iff they have the same \texttt{bid}.


Without \texttt{@generative}, physical equality is under-specified: two
abstractly equal values may or may not be physically equal, meaning CAS
correctness depends on compiler behavior rather than program logic. The identity
becomes explicit through the \texttt{[@generative]} annotation, physical
equality is no longer under-specified for generative blocks, and CAS reasoning
becomes stable and provable. Zoo demonstrates this with the Rcfd (recursive file
descriptor) example where verification fails when the file descriptor state is
non-generative, but succeeds after marking it \texttt{[@generative]}, because
CAS operations on the state can now be relied upon. The intuition is :
\emph{generative} constructors tell the compiler this value has identity ; do
not clone it. This restores the invariant that if CAS fails, it is because the
concurrent state truly changed, not because the compiler duplicated a value.

\subsection{Conclusion}
This paper introduces Zoo, a practical framework for verifying concurrent OCaml
5 programs, bridging real-world code and mechanized verification through
ZooLang. It also contributes language improvements such as atomic record fields
and a precise semantics for physical equality. Using Zoo, the authors verified a
subset of the OCaml standard library, components of the \textsf{Eio} library,
and a large portion of the \textsf{Saturn} lock-free data structure library,
including stacks, queues, bags, and a work-stealing deque. These results
demonstrate the practicality and scalability of the framework. The project
remains actively developed and last semester, I made couple of minor
contributions to the
codebase.\footnote{\url{https://github.com/clef-men/zoo/pull/2}}
\footnote{\url{https://github.com/clef-men/zoo/pull/1}}

\section{Efficient Multi-word Compare and Swap\cite{Guerraoui2013}}\label{sec:mcas}

\subsection{Motivation}

At this stage, the verification landscape includes several automated
linearizability provers and checkers, supporting both stateless and stateful
exploration of concurrent executions.

We now turn to a different flavored data structure: multi-word compare-and-swap
(MCAS), as verified in Zoo~\cite{zoo2026}. Unlike simpler concurrent structures,
MCAS exhibits non-local completion: an operation may be completed by a thread
other than the one that invoked it, a phenomenon also observed in the automatic
linearizability work of Vafeiadis~\cite{Vafeiadis2010}. This helping behavior
complicates reasoning, as the linearization point may lie outside the extent of
the calling thread.

Our goal is therefore to examine the complexities involved in defining and
verifying a data structure such as MCAS, and to understand how it is engineered
for multicore systems. MCAS is specifically designed to exploit fine-grained
concurrency rather than relying on coarse-grained mutual exclusion. Its
correctness relies on descriptor-based indirection, cooperative helping, and
careful synchronization protocols. For this reason, MCAS becomes a compelling
and a necessary case study.


\subsection[Verification Detail]{Proposed Solution}

MCAS generalizes CAS to multiple memory locations. An MCAS operation over $k$
locations atomically performs the following logical action:

\begin{center}
If all locations $x_1, \ldots, x_k$ contain values $o_1, \ldots, o_k$, then
update them to $n_1, \ldots, n_k$; otherwise, leave all locations unchanged.
\end{center}


\subsubsection{Classical Descriptor-Based MCAS}

Traditional lock-free MCAS implementations use a descriptor to coordinate the
update across multiple locations \cite{mcas-harris-2002}. The protocol proceeds
in three phases.

\paragraph{Example}

Consider three memory locations:
\begin{center}
$x = 1, \quad y = 2, \quad z = 3$
\end{center}

We wish to atomically perform:
\begin{center}
$(x, y, z) : (1,2,3) \rightarrow (4,5,6)$
\end{center}
Here, $k = 3$.

\paragraph{Phase 1: Freeze (Installation) Phase :: $k$ CAS operations}

Each target location is updated using CAS from its expected value to a pointer
to a shared descriptor $D$. This CAS both verifies that the location holds the
expected value and installs the descriptor pointer into that word.


\begin{center}
\texttt{CAS}$(x, 1, \&D)$ \\ \texttt{CAS}$(y, 2, \&D)$ \\ \texttt{CAS}$(z, 3,
\&D)$
\end{center}

If any CAS fails, the operation aborts.

\paragraph{Phase 2: Decision Phase :: 1 CAS operation}

The descriptor's status is atomically updated to record the outcome:

\begin{center}
\texttt{CAS}$(D.\text{status}, \text{ACTIVE}, \text{SUCCESS})$
\end{center}

This CAS constitutes the linearization point of the MCAS operation.

\paragraph{Phase 3: Cleanup Phase :: $k$ CAS operations}

Each frozen location is updated based on the descriptor's final state:

\begin{center}
\texttt{CAS}$(x, \&D, 4)$ \\ \texttt{CAS}$(y, \&D, 5)$ \\ \texttt{CAS}$(z, \&D,
6)$
\end{center}


The number of CAS operations are:
\begin{equation}
k \text{ (freeze)} + 1 \text{ (decision)} + k \text{ (cleanup)} = 2k + 1
\end{equation}

For the example above, the result is 7 CAS operations.


\subsection{Efficient MCAS: $(k + 1)$ CAS }


This paper observes that the cleanup phase is conceptually unnecessary: once the
global decision has been made, the final value of each location is already fixed
at the logical level.


The central insight of this paper's protocol is the following:

\begin{center}
\textit{Rather than restoring or overwriting memory locations after the
  decision, encode the final outcome in the descriptor and interpret each
  location's value through the descriptor.}
\end{center}

Consequently, a memory location that still references a descriptor is treated as
logically containing either the old value or the new value, depending solely on
the descriptor's finalized state.

We illustrate the protocol using the same running example:
\begin{center}
$x = 1, \quad y = 2, \quad z = 3$
\end{center}

The intended atomic update is:
\begin{center}
$(x, y, z) : (1,2,3) \rightarrow (4,5,6)$
\end{center}

\paragraph{Phase 1: Freeze Phase :: $k$ CAS Operations}

Each target location is atomically updated from its expected value to a pointer
to a shared descriptor $D$:

\begin{center}
\texttt{CAS}$(x, 1, \&D)$ \\ \texttt{CAS}$(y, 2, \&D)$ \\ \texttt{CAS}$(z, 3,
\&D)$
\end{center}

As in classical MCAS, each CAS simultaneously checks the expected value and
marks the location as participating in the ongoing MCAS.

\paragraph{Phase 2: Decision Phase :: 1 CAS Operation}

The outcome of the operation is finalized by atomically updating the
descriptor's status:

\begin{center}
\texttt{CAS}$(D.\text{status}, \text{ACTIVE}, \text{SUCCESS})$
\end{center}

This operation serves as the linearization point for the MCAS.

\paragraph{Elimination of the Cleanup Phase}

In contrast to classical MCAS, the efficient MCAS \cite{Guerraoui2013} performs
no explicit cleanup CAS's.

\begin{itemize}
\item After the decision phase, memory locations may continue to physically
  reference the descriptor.
\item When a thread reads such a location, it consults the descriptor:
\begin{itemize}
\item If the descriptor state is \texttt{SUCCESS}, the location is interpreted
  as holding the new value.
\item If the state is \texttt{FAILED}, the location is interpreted as holding
  the original value.
\end{itemize}
\item As a result, the memory is logically updated without requiring immediate
  physical rewrites.
\end{itemize}

The total number of CAS operations is therefore:
\begin{equation}
k \text{ (freeze)} + 1 \text{ (decision)} = k + 1
\end{equation}

For the three-location example, this reduces the cost from 7 CAS operations to 4
CAS operations.

\subsection{Implementation Overview}

\begin{figure}[t]
  \centering
  \begin{lstlisting}[language=C,
                      basicstyle=\normalsize\ttfamily, numbers=left,
                      numberstyle=\footnotesize, keywordstyle=\bfseries,
                      commentstyle=\itshape\color{gray}, showstringspaces=false,
                      frame=single, breaklines=true, breakatwhitespace=true]
    readInternal(void *addr, MCASDescriptor *self) { retry_read: val = *addr;

    if (!isDescriptor(val)) { return <val, val>; }

    MCASDescriptor *parent = val->parent;

    if (parent != self && parent->status == ACTIVE) { MCAS(parent); goto
      retry_read; } else { return parent->status == SUCCESSFUL ? <val, val->new>
      : <val, val->old>; } }
  \end{lstlisting}
  \caption{The \texttt{readInternal} auxiliary function: handling descriptors
    and helping.}
  \label{fig:readinternal}
\end{figure}

\begin{figure}[t]
  \centering
  \begin{lstlisting}[language=C,
                      basicstyle=\normalsize\ttfamily, numbers=left,
                      numberstyle=\footnotesize, keywordstyle=\bfseries,
                      commentstyle=\itshape\color{gray}, showstringspaces=false,
                      frame=single, breaklines=true, breakatwhitespace=true]
    read(void *address) { <content, value> = readInternal(address, NULL); return
      value; }

MCAS(MCASDescriptor *desc) { success = true;

  for wordDesc in desc->words { retry_word: <content, value> =
    readInternal(wordDesc.address, desc);

      if (content == &wordDesc) { continue; }

      if (value != wordDesc.old) { success = false; break; }

      if (desc->status != ACTIVE) { break; }

      if (!CAS(wordDesc.address, content, &wordDesc)) { goto retry_word; } }

  if (CAS(&desc.status, ACTIVE, success ? SUCCESSFUL : FAILED)) {
    retireForCleanup(desc); }

  returnValue = (desc.status == SUCCESSFUL); return returnValue; }
  \end{lstlisting}
  \caption{The read and MCAS operations: two-phase protocol (locking and
    finalization).}
  \label{fig:mcas-ops}
\end{figure}

The code in Figures \ref{fig:readinternal} and \ref{fig:mcas-ops} presents a
simplified, essential representation of the Guerraoui et al.\ MCAS algorithm,
omitting memory reclamation details and epoch tracking for clarity. The core
mechanisms of helping, locking, and non-local linearization are fully preserved
in this minimal form.

\paragraph{The \texttt{readInternal} Function}

The \texttt{readInternal} function (Figure \ref{fig:readinternal}) begins by
directly reading the memory location (Line 4). If the location does not contain
a descriptor pointer, it immediately returns the value as-is (Line 6). This happy
path handles the common case where no concurrent MCAS operation has touched the
location. However, if a descriptor pointer is discovered (Line 9), the thread
obtains the parent descriptor and must determine whether that operation is still
active.

The helping mechanism kicks in at Lines 11-14. When a thread encounters an
ACTIVE descriptor whose parent is not its own operation, it recursively calls
\texttt{MCAS(parent)} to drive that operation to completion before proceeding
(Lines 12-13). This ensures that no ACTIVE descriptor can remain indefinitely;
any thread observing it will complete it. The self-check (\texttt{parent !=
  self}) prevents an operation from recursively helping itself. After helping
completes, the thread retries the read to observe the finalized state (Line 14).
If the descriptor is already finalized (either SUCCESSFUL or FAILED), the thread
interprets its result: for successful operations, it returns the new value; for
failed operations, it returns the old value (Lines 16-19). This design ensures
that reads never observe partial states ; they see either the state before the
operation's linearization point or after it.

\paragraph{The \texttt{read} Operation}

The \texttt{read} operation (Figure \ref{fig:mcas-ops}, Lines 1-4) is a simple
wrapper around \texttt{readInternal} with \texttt{self} set to NULL, indicating
that this is not part of an MCAS operation.

\paragraph{The MCAS Operation: Two-Phase Structure}

The MCAS operation exhibits a clear two-phase structure: locking and
finalization.

In the \emph{locking phase} (Figure \ref{fig:mcas-ops}, Lines 6-15), the
operation iterates over each target word in sorted order. For each
word, it first calls \texttt{readInternal} (Line 7) to obtain the current value
while handling any helping obligations. If the word already points to the
current descriptor (Line 9), the thread moves to the next
word indicating that another thread (or a prior attempt in the current
operation) has already locked this location. If the current value does not match
the expected old value (Lines 11), the operation must fail, so the thread
breaks the loop and proceeds to finalization. If the operation status has
changed (Line 13), the loop exits to prevent re-acquiring locations.
Finally, the thread attempts a CAS to install a pointer to the descriptor at the
target location (Lines 15). If the CAS fails, the thread retries (back to
\texttt{retry\_word}), as the failure may indicate that another thread has
concurrently helped this same operation to lock the same word.

Once a location is locked by a descriptor, it
remains locked. When all locations are either already locked or successfully
locked, the operation enters the \emph{finalization phase} (Lines 17-18). Here,
a single CAS atomically changes the descriptor status from ACTIVE to either
SUCCESSFUL (if all locking succeeded) or FAILED (if any value mismatch was
detected). This status CAS is the linearization point (Line 17): it marks
the instant at which the multi-word update becomes logically atomic. The status
field  determines whether subsequent reads
observe the new or old values. Only one thread can successfully perform this CAS
(due to the atomic nature of CAS), ensuring that the operation is finalized
exactly once.

\paragraph{Non-Local Linearization Points}

The non-locality of the linearization point emerges clearly from this structure.
Readers determine the logical value of a location by examining the descriptor
status, which may be modified on a completely different memory location than the
data words themselves. When a read observes a locked location (one pointing to a
descriptor), it must wait for that descriptor's status to be finalized, at which
point it learns whether the associated update succeeded. Thus, the effect of a
k-word MCAS is determined by a single CAS on the descriptor's status field, not
by k separate updates to data.
In the common uncontended case, an MCAS requires
only $k+1$ CASes (one per data word to lock, plus one for the status).

\subsection{Conclusion}
MCAS is therefore included not as an isolated or exceptional case, but as a
deliberate benchmark for expressiveness. Its correctness fundamentally depends
on non-local linearization and helping-driven interference.

\section{Cosmo: a concurrent separation logic for multicore OCaml \cite{cosmo2020}}


\subsection{Motivation}
So far, we have established a framework for verifying concurrent programs under sequential consistency (SC). Under SC, the execution of a concurrent program can be understood as an interleaving of thread-local actions over a single, centralized shared memory, which makes reasoning about correctness comparatively straightforward.

However, real-world programming languages, such as OCaml, do not operate under sequential consistency. Instead, they adopt relaxed (or weak) memory models, where program executions are not necessarily representable as simple interleavings of thread actions. As a result, correctness reasoning must account for the semantics of the underlying memory model before addressing higher-level concurrent behavior.

 To reason about concurrent programs in such languages, one must first reason at
 a low level, directly in terms of the memory model and its allowed reorderings
 and visibility constraints. The terms weak memory model and relaxed memory model are used interchangeably, both referring to the fact that program execution is no longer constrained to behave as if all threads interact with a single, sequentially consistent memory. Instead, memory effects may be delayed, reordered, or observed differently by different threads.

Once a framework for low-level reasoning is in place, it becomes possible to
reason modularly about libraries of concurrent data structures, and subsequently
about the high-level programs that rely on them.

Fortunately, the Multicore OCaml memory model is relatively simple when compared to many industrial weak memory models. It distinguishes between two kinds of memory locations: atomic and nonatomic. Atomic locations come with well-defined synchronization guarantees, whereas nonatomic locations must be accessed in a data-race-free manner to ensure meaningful behavior.

Reasoning about atomic locations follows a similar spirit to the approach discussed in earlier work on linearizability-based verification \cite{Vafeiadis2010}. In particular, when an operation performs a stateful change, it becomes possible to identify a clear correctness condition tied to the update. In contrast, stateless operations require the logic to retain and compose all relevant information to justify correctness.

Cosmo builds precisely such a framework: one that supports low-level reasoning about relaxed memory behaviors while enabling compositional, high-level reasoning about concurrent programs and data structures.

\subsection{Proposed Solution}



\begin{figure}[t]
\centering
\begin{align*}
a &\in \mathsf{Loc}_{\mathsf{NA}} \\
A &\in \mathsf{Loc}_{\mathsf{AT}} \\
t &\in \mathsf{Time} \;\triangleq\; \mathbb{Q} \cap [0,\infty) \\
h &\in \mathsf{Hist} \;\triangleq\; \mathsf{Time} \xrightarrow{\mathit{fin}} \mathsf{Val} \\
V, W, G &\in \mathsf{View} \;\triangleq\; \mathsf{Loc}_{\mathsf{NA}} \xrightarrow{\mathit{set}} \mathsf{Time} \\
\sigma &\in \mathsf{Store} \;\triangleq\;
  (\mathsf{Loc}_{\mathsf{NA}} \xrightarrow{\mathit{fin}} \mathsf{Hist})
  \times
  (\mathsf{Loc}_{\mathsf{AT}} \xrightarrow{\mathit{fin}} (\mathsf{Val} \times \mathsf{View}))
\end{align*}
\caption{Semantic objects: locations, time stamps, histories, views, and stores.}
\label{fig:semantic-objects}
\end{figure}

\begin{figure}[t]
\centering
\begin{mathpar}

\infer[\textsc{Mem-NA-Alloc}]
{
  a \notin \mathrm{dom}\,\sigma
  \quad
  h = \{0 \mapsto v\}
}
{
  \sigma; W \xrightarrow{\mathrm{alloc}(a,v)} \sigma[a \mapsto h]; W
}

\infer[\textsc{Mem-AT-Alloc}]
{
  A \notin \mathrm{dom}\,\sigma
}
{
  \sigma; W \xrightarrow{\mathrm{alloc}(A,v)} \sigma[A \mapsto (v,W)]; W
}

\infer[\textsc{Mem-NA-Read}]
{
  h = \sigma(a)
  \quad
  t \in \mathrm{dom}\,h
  \quad
  W(a) \le t
  \quad
  v = h(t)
}
{
  \sigma; W \xrightarrow{\mathrm{rd}(a,v)} \sigma; W
}

\infer[\textsc{Mem-AT-Read}]
{
  \sigma(A) = (v,V)
}
{
  \sigma; W \xrightarrow{\mathrm{rd}(A,v)} \sigma; W \cup V
}

\infer[\textsc{Mem-NA-Write}]
{
  h = \sigma(a)
  \quad
  t \notin \mathrm{dom}\,h
  \quad
  W(a) < t
  \quad
  h' = h[t \mapsto v]
}
{
  \sigma; W \xrightarrow{\mathrm{wr}(a,v)} \sigma[a \mapsto h']; W[a \mapsto t]
}

\infer[\textsc{Mem-AT-Write}]
{
  \sigma(A) = (v,V)
  \quad
  V' = W' = W \cup V
}
{
  \sigma; W \xrightarrow{\mathrm{wr}(A,v')} \sigma[A \mapsto (v',V')]; W'
}

\infer[\textsc{Mem-AT-Read-Write}]
{
  \sigma(A) = (v,V)
  \quad
  V' = W' = W \cup V
}
{
  \sigma; W \xrightarrow{\mathrm{rdwr}(A,v,v')} \sigma[A \mapsto (v',V')]; W'
}

\end{mathpar}
\caption{Operational semantics for memory actions over nonatomic and atomic locations.}
\label{fig:memory-semantics}
\end{figure}




Cosmo makes the underlying relaxed memory model explicit using a small collection of semantic objects: locations, timestamps, histories, views, and stores. Non-atomic locations are associated with histories (h) that record write events as timestamp–value pairs, while atomic locations store both a value and an associated view (V). A view is a mapping from non-atomic locations to timestamps and represents the latest writes that are visible. Each thread carries its own view (W), which records what that thread is currently allowed to observe. The store $\sigma$ aggregates these components, enabling the logic to precisely describe how writes are created, ordered, and observed, and to distinguish between operations that merely consult visibility and those that propagate it.

For the purposes of this report, to maintain brevity and to pepper my understanding of the paper while building intuition, we adopt a Manhattan-style city-grid analogy. \footnote{The ``Manhattan'' grid analogy and a literature that uses \emph{atomics} quite frequently are purely coincidental, despite the historical temptation.}
Each memory location is a building whose floors correspond to writes over time. The view (W) is the photograph of the city carried by the active thread: it may be incomplete and merely inspecting a building using this photograph does not update it. The view (V) is a photograph stored inside an atomic building, left behind by an atomic write and retrievable by other threads. Non-atomic reads and writes consult only (W), do not store or propagate photographs, and may therefore observe writes that are not globally latest but are nonetheless visible according to the thread’s current snapshot. Atomic operations, in contrast, merge (W) with (V), explicitly propagating visibility. Allowing such stale non-atomic reads is not a weakness of the model; it faithfully captures the fact that, under relaxed memory, threads only know what their current view (W) entitles them to know, and synchronization must be made explicit through atomic operations.

\subsection*{BaseCosmo}

The operational semantics of Cosmo describe how a program executes at the machine level: an expression (e) reduces to a new expression (e'), possibly interacting with the memory subsystem through a memory event and possibly spawning new threads. These semantics precisely track how memory locations, histories, timestamps, and views evolve during execution. However, while such a semantics tells us *what can happen*, it does not by itself tell us *how to reason compositionally* about programs, or how to prove that a program is safe or correct.

Cosmo instantiates Iris, a generic framework for building concurrent separation logics. Iris provides the logical infrastructure, assertions, ghost state, invariants, weakest preconditions, and proof rules, while BaseCosmo supplies the Multicore OCaml–specific ingredients.

At this point, the Manhattan city-grid analogy becomes useful again. In the operational model, each memory location is a building whose floors are writes, histories record all floors ever built, and views determine how tall each building appears to a given thread. BaseCosmo’s role is to lift this concrete picture into logic. Instead of directly manipulating buildings and photographs, the logic reasons about \emph{ownership} and \emph{knowledge}. Fractional permissions allow ownership of buildings to be split across threads, while a global view invariant ensures that every thread’s snapshot (W) remains consistent with the actual city skyline represented by the store. In other words, BaseCosmo ensures that the logical “photographs” carried by proofs faithfully correspond to the operational photographs carried by threads.

To make this precise, BaseCosmo introduces assertions such as points-to assertions and a valid-view assertion. Points-to assertions describe ownership and knowledge of individual memory locations both nonatomic and atomic, while the valid-view assertion captures the global invariant relating thread views to the store. Atomic points-to assertions additionally account for the fact that atomic locations store a view (V), which can later be merged into a thread’s view (W), explicitly propagating visibility.

Hoare triples are the interface through which all of this reasoning is exposed. A triple ({P}, (e, W), {$\Phi$}) states that if the precondition (P) holds, then executing expression (e) in a thread with view (W) is safe, and if it terminates, it produces a value and an updated view satisfying the postcondition ($\Phi$). Iris supplies the generic machinery for weakest preconditions, framing, and invariants, while BaseCosmo provides the memory-specific axioms that explain how reads, writes, allocations, and atomic operations affect ownership and views.



\subsection*{A Higher Level Logic : Cosmo}

While BaseCosmo provides a faithful translation of the Multicore OCaml operational semantics, it remains intentionally low-level. In particular, it exposes details such as histories, timestamps, and per-thread views directly in assertions. This precision is useful for soundness, but it makes reasoning cumbersome: even in data-race-free code, a nonatomic location appears to store an entire history rather than a single value, and every Hoare triple must explicitly mention the current thread’s view (W).

Finally, all the preamble data takes place. Its goal is not to change the underlying semantics, but to provide a more convenient interface for reasoning. On top of BaseCosmo, Cosmo offers a higher-level logic in which nonatomic locations can again be reasoned about as if they store a single value, provided the program is data-race free while still remaining sound with respect to relaxed memory.

Back to our analogy - at the BaseCosmo level, reasoning involves explicit references to building histories and photographs:  which floors exist, which timestamps label them, and which buildings are visible in a thread’s snapshot. Cosmo abstracts over these details. From the perspective of Cosmo, a nonatomic building simply “contains a value,” meaning that the most recent write to that building is known to the current thread.

Instead of asserting a BaseCosmo proposition directly, a Cosmo assertion denotes a function from views to BaseCosmo assertions. Subjective assertions, such as “this thread has seen view (V),” depend on the current thread’s snapshot and therefore cannot be shared. Objective assertions, on the other hand, are independent of any particular thread’s view and can be placed inside invariants. This distinction explains why nonatomic points-to assertions in Cosmo are necessarily subjective: they assert not only that a value exists, but that the current thread is aware of the corresponding write.

Iris continues to play a central role in this construction. It provides the general machinery for weakest preconditions, Hoare triples, framing, and invariants, while Cosmo refines the assertion language to separate subjective, view-dependent knowledge from objective, shareable facts. Hoare triples in Cosmo therefore look familiar: a precondition (P) describes the resources required to execute an expression (e), and the postcondition \(\Phi\) describes the resources obtained after execution. What changes is that Cosmo carefully controls how view-dependent information flows through these triples, ensuring monotonicity with respect to view growth and preserving soundness under relaxed memory.



\subsection{Conclusion}

This paper introduced Cosmo, a concurrent separation logic that enables formal reasoning about Multicore OCaml under its relaxed memory model. Unlike traditional program logics that assume sequential consistency, Cosmo is designed to reason directly about the realities of modern multicore execution, including delayed visibility, partial views, and explicit synchronization.

The broader goal of this paper was also to understand how the verification of concurrent programs evolve as we move closer to real systems. While Cosmo significantly advances the state of the art by enabling formal reasoning about Multicore OCaml under a relaxed memory model, it is not the end of the story. As indicated by the authors, several important directions remain open, including support for arrays, richer synchronization primitives such as the Domain API, and full verification of sophisticated lock-free data structures. More fundamentally, reasoning about programs that intentionally exploit data races on nonatomic locations remains an open challenge.

\chapter{Paper 6: Paper Title Here}
\label{ch:paper6}

\section*{Paper Information}
\begin{itemize}
    \item \textbf{Title:} Full title of the paper
    \item \textbf{Authors:} Author names
    \item \textbf{Venue:} Journal or Conference name
    \item \textbf{Year:} Publication year
    \item \textbf{Citation:} \cite{paper6}
\end{itemize}

\vspace{1cm}

\section{Problem Statement and Motivation}
\subsection{Research Problem}
\textit{Add your analysis here...}

\subsection{Motivation}
\textit{Add your analysis here...}

\section{Main Contributions}
\begin{enumerate}
    \item \textbf{Contribution 1:} Description
    \item \textbf{Contribution 2:} Description
    \item \textbf{Contribution 3:} Description
\end{enumerate}
\textit{Add detailed explanation...}

\section{Technical Approach}
\subsection{Methodology}
\textit{Add your analysis here...}

\subsection{Key Techniques}
\textit{Add detailed technical analysis...}

\subsection{Related Work}
\textit{Add analysis of related work...}

\section{Evaluation}
\subsection{Experimental Setup}
\textit{Add experimental setup details...}

\subsection{Results}
\textit{Add results summary...}

\subsection{Result Analysis}
\textit{Add analysis of results...}

\section{Strengths}
\begin{itemize}
    \item Strength 1
    \item Strength 2
    \item Strength 3
\end{itemize}
\textit{Add detailed assessment...}

\section{Limitations}
\begin{itemize}
    \item Limitation 1
    \item Limitation 2
    \item Limitation 3
\end{itemize}
\textit{Add detailed assessment...}

\section{Significance and Impact}
\subsection{Significance to the Field}
\textit{Add assessment...}

\subsection{Practical Applications}
\textit{Add discussion...}

\subsection{Future Research Directions}
\textit{Add discussion...}

\section{Summary}
\textit{Add summary...}

\section{Critical Questions}
\begin{enumerate}
    \item Question 1
    \item Question 2
    \item Question 3
\end{enumerate}
\textit{Add critical analysis...}

\newpage



\bibliographystyle{plain}
\bibliography{./references}
\end{document}
